\documentclass{article}
\usepackage[utf8]{inputenc}
\usepackage[LFE,LAE,T2A,T1]{fontenc}
\usepackage[english,arabic,greek,russian,polish]{babel}
\usepackage{arabtex} % Required for arab
\usepackage{CJKutf8} % Required for Chinese, Japanese and Korean
\newcommand{\textcn}[1]{\begin{CJK}{UTF8}{gbsn}#1\end{CJK}}
\newcommand{\textjp}[1]{\begin{CJK}{UTF8}{min}#1\end{CJK}}
\newcommand{\textko}[1]{\begin{CJK}{UTF8}{mj}#1\end{CJK}}

\title{\LaTeX~Unicode Template}
\author{Mauricio Matamoros}

\begin{document}
\selectlanguage{english}% Set English as default
\maketitle

\begin{abstract}
Writing simultaneously in several languages in \LaTeX{} can produce headaches.
Nonetheless, when updated and properly set, it becomes pretty straightforward.
The present document explains how to write in (at least) 13 different languages other than English by adding less than 10 lines of code.
Of course, compatibility with all packages is not guaranteed.
\end{abstract}

\section{Example}
\begin{center}
\begin{tabular}{c c}
Arabic     & \foreignlanguage{arabic}{هذا عربي} \\
Chinese    & \textcn{这是汉语} \\
French     & C'est français \\
German     & Einige große Wörter auf Deutsch \\
Greek      & \foreignlanguage{greek}{αυτό είναι ελληνικό} \\
Japanese   & \textjp{これは日本語です} \\
Korean     & \textko{이것은 한국이다.} \\
Norwegian  & Norsk er søtt \\
Polish     & Niektóre słowa po polsku \\
Portuguese & Português não é espanhol \\
Russian    & \foreignlanguage{russian}{Это русский} \\
Spanish    & Esto es español \\
Swedish    & Några ord på svenska \\
\end{tabular}
\end{center}

\section{Making it work}
Import the following packages right after defining the document class

\subsection{Snippet}
\begin{verbatim}
\usepackage[utf8]{inputenc}
\usepackage[LFE,LAE,T2A,T1]{fontenc}
\usepackage[english,arabic,greek,russian,polish]{babel}
\usepackage{arabtex} % Required for arab
\usepackage{CJKutf8} % Required for Chinese, Japanese and Korean
\end{verbatim}

Just copy-pasting these five lines will do most of the work.

\subsection{Packages in detail}
\begin{enumerate}
	\item \textbf{Input encoding set to utf8:} The source must be encoded using the utf-8 character set, and so it should be told to \LaTeX.

	\item \textbf{Set the proper font encoding:} To produce the correct output, all unicode pages in Computer Modern must be enabled.
	We do so by adding the following options to \texttt{fontenc}.
	\begin{itemize}
		\item[\texttt{T1}] The extended Latin set is accessible by enabling \texttt{T1}. This will give us access to virtually all western-European languages like French, German, Italian, Norwegian, Spanish, Swedish, Portuguese, 
		\item[\texttt{T2A}] This little guy gives us access to the Cyrillic pages of the font. Useful for Russian (at least).
		\item[\texttt{LFE,LAE}] These two fellas allow us to write in Arab.
	\end{itemize}

	\item \textbf{Add babel with required languages:} Arabic, Greek, Russian, and Polish need to be defined in Babel for this example to work.
	Here it is important to remark that the \texttt{\textbackslash{}foreignlanguage} macro is necessary for writing in Arabic and Greek, but not in Cyrillic.
	However, Greek and Russian are mutually exclusive without Babel, so using them together makes necessary to wrap both inputs with the aforementioned macro. 
\end{enumerate}

\end{document} 